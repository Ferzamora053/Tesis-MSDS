%%%%%%%%%%%%%%%%%%%%%%%%%%%%%%%%%%%%%%%%
% Masters/Doctoral Thesis 
% LaTeX Template
% Version 2.5 (27/8/17)
%
% This template was downloaded from:
% http://www.LaTeXTemplates.com
%
% Version 2.x major modifications by:
% Vel (vel@latextemplates.com)
%
% This template is based on a template by:
% Steve Gunn (http://users.ecs.soton.ac.uk/srg/softwaretools/document/templates/)
% Sunil Patel (http://www.sunilpatel.co.uk/thesis-template/)
%
% Template license:
% CC BY-NC-SA 3.0 (http://creativecommons.org/licenses/by-nc-sa/3.0/)
%
%%%%%%%%%%%%%%%%%%%%%%%%%%%%%%%%%%%%%%%%%

%----------------------------------------------------------------------------------------
%	PACKAGES AND OTHER DOCUMENT CONFIGURATIONS
%----------------------------------------------------------------------------------------

\documentclass[
11pt, % The default document font size, options: 10pt, 11pt, 12pt
oneside, % Two side (alternating margins) for binding by default, uncomment to switch to one side
spanish, % ngerman for German
singlespacing, % Single line spacing, alternatives: onehalfspacing or doublespacing
%draft, % Uncomment to enable draft mode (no pictures, no links, overfull hboxes indicated)
%nolistspacing, % If the document is onehalfspacing or doublespacing, uncomment this to set spacing in lists to single
%liststotoc, % Uncomment to add the list of figures/tables/eta, haberc to the table of contents
%toctotoc, % Uncomment to add the main table of contents to the table of contents
parskip, % Uncomment to add space between paragraphs
%nohyperref, % Uncomment to not load the hyperref package
headsepline, % Uncomment to get a line under the header
%chapterinoneline, % Uncomment to place the chapter title next to the number on one line
%consistentlayout, % Uncomment to change the layout of the declaration, abstract and acknowledgements pages to match the default layout
table
]{MastersDoctoralThesis} % The class file specifying the document structure


\usepackage{xcolor}

\usepackage{soul}
\usepackage{graphicx}
%\usepackage{subfig}


\usepackage[bottom]{footmisc}

\usepackage[utf8]{inputenc} % Required for inputting international characters
%\usepackage[T1]{fontenc} % Output font encoding for international characters

\usepackage{mathpazo} % Use the Palatino font by default
\usepackage{amsmath}
\usepackage{caption}

\usepackage{float}

\usepackage{hyperref}
\usepackage{apacite} %EN CASO DE NO USAR APA STYLE COMENTAR ESTA LINEA Y CAMBIAR EL ESTILO DE ABAJO
% \usepackage{cite}

% \usepackage{subcaption}

%\usepackage[backend=bibtex, natbib=true]{biblatex} % Use the bibtex backend with the authoryear citation style (which resembles APA)
% SI AGREGO STYLE=AUTHORYEAR PARA QUE SEA ESTILO APA, SE ME DESORDENA EL FORMATO DE LOS AUTORES EN LA BIBLIOGRAFIA... MUY RARO, NO ENCUENTRO SOLUCION EN GOOGLE HASTA EL MOMENTO
%\usepackage[
%backend=biber,
%style=alphabetic,
%citestyle=authoryear
%]{biblatex}

% Use the bibtex backend with the authoryear citation style (which resembles APA)
%,style=authoryear,natbib=true

% \addbibresource{citas.bib} % The filename of the bibliography
\usepackage[autostyle=true]{csquotes} % Required to generate language-dependent quotes in the bibliography
\usepackage{color}
%----------------------------------------------------------------------------------------
%	MARGIN SETTINGS
%----------------------------------------------------------------------------------------

\geometry{
	paper=a4paper, % Change to letterpaper for US letter
	inner=2.5cm, % Inner margin
	outer=3.8cm, % Outer margin
	bindingoffset=.5cm, % Binding offset
	top=1.5cm, % Top margin
	bottom=1.5cm, % Bottom margin
	%showframe, % Uncomment to show how the type block is set on the page
}

%----------------------------------------------------------------------------------------
%	THESIS INFORMATION
%----------------------------------------------------------------------------------------

\thesistitle{Metodología de generación de \enquote{super-categorías} con modelos de lenguaje grande (LLM) para la mitigación del problema del \enquote{cold-start} en sistemas de recomendación.} % Your thesis title, this is used in the title and abstract, print it elsewhere with \ttitle
\thesistitleEnglish{Methodology for generating \enquote{super-categories} with large language models (LLM) to mitigate the \enquote{cold-start} problem in recommendation systems.} % Your thesis title, this is used in the title and abstract, print it elsewhere with \ttitleEnglish
\supervisor{Claudio Esteban Díaz Cifuentes\\EN CASO DE CO-GUIA} % Your supervisor's name, this is used in the title page, print it elsewhere with \supname
\examiner{NOMBRE COMISON\\2DO NOMBRE COMISION} % Your examiner's name, this is not currently used anywhere in the template, print it elsewhere with \examname
% \degree{comente esta línea y descomente la linea de degree correspondiente}
% \degree{Magíster en Ciencias de la Ingeniería, mención ESCRIBIR EL NOMBRE DE LA MENCION} % Your degree name, this is used in the title page and abstract, print it elsewhere with \degreename
% \degree{Magíster en Ingeniería Industrial e Investiagación de Operaciones} % Your degree name, this is used in the title page and abstract, print it elsewhere with \degreename
\degree{Master of Science in Data Science; Ingeniería Civil Informática} % Your degree name, this is used in the title page and abstract, print it elsewhere with \degreename
% \degreeEnglish{comment this line and choose the right degree}
% \degreeEnglish{Master of Science in Engineering, concentration XXXX}
\degreeEnglish{Master of Science in Data Science}
% \degreeEnglish{Master in Industrial Engineering and Operations Research}
\author{Fernando Andrés Zamora Carrasco} % Your name, this is used in the title page and abstract, print it elsewhere with \authorname
\addresses{} % Your address, this is not currently used anywhere in the template, print it elsewhere with \addressname
\subject{Machine Learning} % Your subject area, this is not currently used anywhere in the template, print it elsewhere with \subjectname
\keywords{machine learning, marketing, sistemas de recomendación} % Keywords for your thesis, this is not currently used anywhere in the template, print it elsewhere with \keywordnames
\university{\href{https://www.uai.cl/}{Universidad Adolfo Ibañez}} % Your university's name and URL, this is used in the title page and abstract, print it elsewhere with \univname
\department{\href{https://ingenieria.uai.cl/}{Departamento de Ciencias de Datos}} % Your department's name and URL, this is used in the title page and abstract, print it elsewhere with \deptname
\faculty{\href{https://ingenieria.uai.cl/}{Facultad de Ingeniería y Ciencias}} % Your faculty's name and URL, this is used in the title page and abstract, print it elsewhere with \facname
\facultyEnglish{\href{https://ingenieria.uai.cl/}{Faculty of Engineering and Science}} % Your faculty's name and URL, this is used in the title page and abstract, print it elsewhere with \facname

\AtBeginDocument{
\hypersetup{pdftitle=\ttitle} % Set the PDF's title to your title
\hypersetup{pdfauthor=\authorname} % Set the PDF's author to your name
\hypersetup{pdfkeywords=\keywordnames} % Set the PDF's keywords to your keywords
}

\begin{document}

\frontmatter % Use roman page numbering style (i, ii, iii, iv...) for the pre-content pages

\pagestyle{plain} % Default to the plain heading style until the thesis style is called for the body content

%----------------------------------------------------------------------------------------
%	TITLE PAGE
%----------------------------------------------------------------------------------------

\begin{titlepage}
\begin{center}

\vspace*{.06\textheight}
{\scshape\LARGE \univname\par}\vspace{1.5cm} % University name
\textsc{\Large Tesis de Magíster}\\[0.5cm] % Thesis type

\HRule \\[0.4cm] % Horizontal line
{\LARGE \bfseries \ttitle\par}\vspace{0.4cm} % Thesis title
\HRule \\[1.5cm] % Horizontal line
 
\begin{minipage}[t]{0.4\textwidth}
\begin{flushleft} \large
\emph{Autor:}\\
{\authorname} % Author name - remove the \href bracket to remove the link
\end{flushleft}
\end{minipage}
\begin{minipage}[t]{0.4\textwidth}
\begin{flushright} \large
\emph{Profesores guía:} \\
\supname\\
\vspace{5mm}
\emph{Comité de defensa:} \\
\examname\\
%\href{https://ingenieria.uai.cl/profesor/paginaWeb1/}{\supname} \\% Supervisor name - remove the \href bracket to remove the link  
%\href{https://ingenieria.uai.cl/profesor/paginaWeb2/}{\examname} % Supervisor name - remove the \href bracket to remove the link  
\end{flushright}
\end{minipage}\\[2.5cm]
 
\vfill

\large \textit{Tesis realizada acorde a los requerimientos para la obtención de los grados/títulos de\\ \textbf{\degreename}}\\[0.3cm] % University requirement text
\textit{de la}\\[0.4cm]
\facname\\[1.5cm] % Research group name and department name
 
\vfill

{\large \today}\\[2cm] % Date
\includegraphics[width=5cm]{fic.png} % University/department logo - uncomment to place it
 
\vfill
\end{center}
\end{titlepage}

%----------------------------------------------------------------------------------------
%	ABSTRACT PAGE
%----------------------------------------------------------------------------------------

\begin{resumen}
\addchaptertocentry{\abstractname} % Add the abstract to the table of contents

Resuma en no más de 1 página el contenido de su propuesta. Como idea general, debería tener un párrafo por cada uno de los puntos (capítulos) que cubre el trabajo.

Sed ullamcorper quam eu nisl interdum at interdum enim egestas. Aliquam placerat justo sed lectus lobortis ut porta nisl porttitor. Vestibulum mi dolor, lacinia molestie gravida at, tempus vitae ligula. Donec eget quam sapien, in viverra eros. Donec pellentesque justo a massa fringilla non vestibulum metus vestibulum. Vestibulum in orci quis felis tempor lacinia. Vivamus ornare ultrices facilisis. Ut hendrerit volutpat vulputate. Morbi condimentum venenatis augue, id porta ipsum vulputate in. Curabitur luctus tempus justo. Vestibulum risus lectus, adipiscing nec condimentum quis, condimentum nec nisl. Aliquam dictum sagittis velit sed iaculis. Morbi tristique augue sit amet nulla pulvinar id facilisis ligula mollis. Nam elit libero, tincidunt ut aliquam at, molestie in quam. Aenean rhoncus vehicula hendrerit.

Sed ullamcorper quam eu nisl interdum at interdum enim egestas. Aliquam placerat justo sed lectus lobortis ut porta nisl porttitor. Vestibulum mi dolor, lacinia molestie gravida at, tempus vitae ligula. Donec eget quam sapien, in viverra eros. Donec pellentesque justo a massa fringilla non vestibulum metus vestibulum. Vestibulum in orci quis felis tempor lacinia. Vivamus ornare ultrices facilisis. Ut hendrerit volutpat vulputate. Morbi condimentum venenatis augue, id porta ipsum vulputate in. Curabitur luctus tempus justo. Vestibulum risus lectus, adipiscing nec condimentum quis, condimentum nec nisl. Aliquam dictum sagittis velit sed iaculis. Morbi tristique augue sit amet nulla pulvinar id facilisis ligula mollis. Nam elit libero, tincidunt ut aliquam at, molestie in quam. Aenean rhoncus vehicula hendrerit.

\textbf{Palabras clave:} Sistemas de Recomendación, Problema del Arranque en Frío, Modelos de Lenguaje Grande, Transferencia entre Dominios.
\end{resumen}

\begin{abstract}
\addchaptertocentry{Abstract} % Add the abstract to the table of contents

IN ENGLISH

\textbf{Keywords:} Recommender Systems, Cold-Start Problem, Large Language Models, Cross-Domain Transfer.

\end{abstract}

%----------------------------------------------------------------------------------------
%	ACKNOWLEDGEMENTS
%----------------------------------------------------------------------------------------

%\begin{acknowledgements}
%\addchaptertocentry{\acknowledgementname} % Add the acknowledgements to the table of contents
%The acknreturnowledgments and the people to thank go here, don't forget to include your project advisor\ldots
%\end{acknowledgements}

%----------------------------------------------------------------------------------------
%	LIST OF CONTENTS/FIGURES/TABLES PAGES
%----------------------------------------------------------------------------------------

\tableofcontents % Prints the main table of contents

%\listoffigures % Prints the list of figures

%\listoftables % Prints the list of tables

%----------------------------------------------------------------------------------------
%	ABBREVIATIONS
%----------------------------------------------------------------------------------------

%\begin{abbreviations}{ll} % Include a list of abbreviations (a table of two columns)

%\textbf{ACHS} & \textbf{A}sociación \textbf{C}hilena \textbf{D}e \textbf{S}eguridad\\
%\textbf{SUSESO} & \textbf{S}uperintendencia \textbf{D}e \textbf{S}eguridad \textbf{S}ocial\\
%\textbf{MUSEG} & \textbf{M}utual \textbf{D}e \textbf{S}eguridad \textbf{D}e \textbf{L}a \textbf{C}ámara \textbf{C}hilena \textbf{D}e \textbf{L}a  \textbf{C}onstrucción \\

%\end{abbreviations}

%----------------------------------------------------------------------------------------
%	THESIS CONTENT - CHAPTERS
%----------------------------------------------------------------------------------------

\mainmatter % Begin numeric (1,2,3...) page numbering

\pagestyle{thesis} % Return the page headers back to the "thesis" style

% Include the chapters of the thesis as separate files from the Chapters folder
% Uncomment the lines as you write the chapters

% Chapter 1
\chapter{Introducción} % Main chapter title
\label{sec:Introduccion} % For referencing the chapter elsewhere, use

\textbf{OMITIR ESTA SECCION PARA EL CASO DEL MAGISTER EN CIENCIAS DE LA INGENIER\'IA}

De forma de acotar la propuesta lo máximo posible, este capítulo debería responder contener al menos los siguientes puntos:
\begin{itemize}
\item Una breve descripción del lugar donde está generada la oportunidad.
\item El contexto u oportunidad de investigación.
\item La importancia y motivación de la investigación y cómo su trabajo es relevante.
\end{itemize}

\newpage
Los sistemas de recomendación tienen una gran importancia en la actualidad, puesto que permiten personalizar la experiencia del usuario en diversas plataformas digitales. 

Estos sistemas utilizan como entrada grandes volúmenes de datos generados por las interacciones de los usuarios con los productos o servicios ofrecidos. A través del análisis de estos datos, los sistemas de recomendación pueden identificar patrones y preferencias individuales, lo que les permite sugerir productos o servicios que se ajusten a los intereses específicos de cada usuario.

Entonces, ¿qué sucede cuando los datos son escasos o inexistentes? Este es el problema conocido como \enquote{cold start} o \enquote{arranque en frío}. En tales situaciones, los sistemas de recomendación enfrentan dificultades para generar sugerencias precisas debido a la falta de información suficiente sobre las preferencias del usuario o las características del producto.

% Este desafío no es meramente teórico; es particularmente agudo en grandes conglomerados de retail que operan con múltiples unidades de negocio (como tiendas por departamento, mejoramiento del hogar o supermercados). En este escenario, un cliente puede tener un amplio historial de compras en una unidad, pero ser completamente "nuevo" para otra. Esta desconexión, causada por catálogos de productos que operan en silos, se traduce en una pérdida significativa de oportunidad comercial y en una experiencia de usuario fragmentada. Por lo tanto, encontrar una forma escalable de transferir el conocimiento del cliente entre estos dominios (unidades de negocio) es una de las principales motivaciones para la industria y la investigación.

Hablar sobre que se ha hecho para mitigar este problema, y que aún existen desafíos abiertos en el área. Utilizar citas de papers.

Por lo tanto, el objetivo de esta tesis es investigar y desarrollar una metodología que permita mitigar el problema del \enquote{cold start} en sistemas de recomendación, utilizando técnicas de dominios de información interrelacionados mediante la generación de \enquote{super-categorías} con modelos de lenguaje grande (LLM) para la vinculación de productos entre distintas unidades de negocio del grupo Falabella.

Para ello, se trabajará en conjunto con la empresa Falabella Tecnología Corporativa (FTC), que cuenta con una amplia base de datos de usuarios y productos en sus plataformas de comercio electrónico, así como distintas unidades de negocio que pueden proporcionar información complementaria para abordar el problema del \enquote{cold start} \cite{L__2012}.
% Chapter 1
\chapter{Definición del problema u oportunidad} % Main chapter title
\label{sec:problema} % For referencing the chapter elsewhere, use

Este capítulo debe contener la exposición general del problema. Con respecto al problema se debe responder las siguientes preguntas:
\begin{itemize}
	\item	¿Cuál es el problema u oportunidad abordado por el proyecto? ¿Es posible cuantificarlo?
	\item	¿Cuáles son las causas de la existencia de este problema u oportunidad? Haga referencia a publicaciones y/u otros antecedentes que validen estas causas.
	\item Explique cómo la memoria ayudará a abordar este problema.
	
\end{itemize}

\section{Ejemplos de latex}

Sed ullamcorper quam eu nisl interdum at interdum enim egestas. Aliquam placerat justo sed lectus lobortis ut porta nisl porttitor. Vestibulum mi dolor, lacinia molestie gravida at, tempus vitae ligula. Donec eget quam sapien, in viverra eros. Donec pellentesque justo a massa fringilla non vestibulum metus vestibulum. Vestibulum in orci quis felis tempor lacinia. Vivamus ornare ultrices facilisis. Ut hendrerit volutpat vulputate. Morbi condimentum venenatis augue, id porta ipsum vulputate in. Curabitur luctus tempus justo. Vestibulum risus lectus, adipiscing nec condimentum quis, condimentum nec nisl. Aliquam dictum sagittis velit sed iaculis. Morbi tristique augue sit amet nulla pulvinar id facilisis ligula mollis. Nam elit libero, tincidunt ut aliquam at, molestie in quam. Aenean rhoncus vehicula hendrerit.

\begin{figure}[th]
	\centering
	\includegraphics[width=10cm]{Figures/Ejemplo}
	\caption{Comparación de la participación de mercado entre mutuales, a enero 2020.}
	\label{fig:Ejemplo}
\end{figure}

% Citando Figura~\ref{fig:Ejemplo} de la sección~\ref{sec:problema}

Lista de items
\begin{itemize}
	\item \textbf{Componente A:} Sed ullamcorper quam eu nisl interdum at interdum enim egestas. Aliquam placerat justo sed lectus lobortis ut porta nisl porttitor. Vestibulum mi dolor, lacinia molestie gravida at, tempus vitae ligula. Donec eget quam sapien, in viverra eros. Donec pellentesque justo a massa fringilla non vestibulum metus vestibulum. Vestibulum in orci quis felis tempor lacinia. Vivamus ornare ultrices facilisis. Ut hendrerit volutpat vulputate. Morbi condimentum venenatis augue, id porta ipsum vulputate in.
	\item \textbf{Componente B:} Sed ullamcorper quam eu nisl interdum at interdum enim egestas. Aliquam placerat justo sed lectus lobortis ut porta nisl porttitor. Vestibulum mi dolor, lacinia molestie gravida at, tempus vitae ligula. Donec eget quam sapien, in viverra eros. Donec pellentesque justo a massa fringilla non vestibulum metus vestibulum. Vestibulum in orci quis felis tempor lacinia. Vivamus ornare ultrices facilisis. Ut hendrerit volutpat vulputate. Morbi condimentum venenatis augue, id porta ipsum vulputate in.
\end{itemize}

Lista enumerada
\begin{enumerate}
	\item \textbf{Componente A:} Sed ullamcorper quam eu nisl interdum at interdum enim egestas. Aliquam placerat justo sed lectus lobortis ut porta nisl porttitor. Vestibulum mi dolor, lacinia molestie gravida at, tempus vitae ligula. Donec eget quam sapien, in viverra eros. Donec pellentesque justo a massa fringilla non vestibulum metus vestibulum. Vestibulum in orci quis felis tempor lacinia. Vivamus ornare ultrices facilisis. Ut hendrerit volutpat vulputate. Morbi condimentum venenatis augue, id porta ipsum vulputate in.
	\item \textbf{Componente B:} Sed ullamcorper quam eu nisl interdum at interdum enim egestas. Aliquam placerat justo sed lectus lobortis ut porta nisl porttitor. Vestibulum mi dolor, lacinia molestie gravida at, tempus vitae ligula. Donec eget quam sapien, in viverra eros. Donec pellentesque justo a massa fringilla non vestibulum metus vestibulum. Vestibulum in orci quis felis tempor lacinia. Vivamus ornare ultrices facilisis. Ut hendrerit volutpat vulputate. Morbi condimentum venenatis augue, id porta ipsum vulputate in.
\end{enumerate}

% Citas a libro \cite{ejemploLibro} o paper~\cite{ejemploPaper}. Usted puede agregar otros tipos como journal o techincal report
\newpage

Falabella Tecnología Corporativa (FTC) actualmente posee un sistema de recomendación (Implicit Collaborative Filtering - Matrix Factorization) desarrollado en BigQuery ML (BQML) que presenta varias limitaciones importantes para el negocio: es demasiado simple para abordar la complejidad del comportamiento del cliente, tiene un bajo nivel de personalización y utiliza únicamente el identificador del cliente como dato de entrada principal, desaprovechando datos relevantes como características demográficas, comportamientos históricos y segmentaciones internas. Estas limitaciones restrigen la capacidad del sistema para generar recomendaciones personalizadas que maximicen la efectividad de las campañas de marketing.

Voy a poner las preguntas que deben responderse en esta sección para que quede más claro a que apunta cada párrafo, luego las eliminaré.

¿Cuál es el problema u oportunidad abordado por el proyecto?
De este modo, en conjunto con FTC, se busca desarrollar un sistema de recomendación basado en dominios cruzados (CDRS) que utilice los datos de las diferentes unidades de negocio del grupo Falabella, con el objetivo de mitigar el problema del \enquote{cold-start} para clientes nuevos de una determinada unidad de negocio, es decir, aquellos clientes que no cuentan con un historial previo de interacciones en dicha unidad, pero que sí poseen datos relevantes en las demás unidades de negocio del grupo. 

¿Se puede cuantificar?
Sí, la cuantificación del problema se asocia a la siguiente figura, donde se muestra principalmente, el costo de oportunidad que significaría aumentar el pool/número de clientes potenciales que se podrían captar si se implementara un sistema de recomendación basado en dominios cruzados (CDRS) que utilice datos de las diferentes unidades de negocio del grupo Falabella.

\begin{figure}[th]
	\centering
	\includegraphics[width=\textwidth]{Figures/grafica Matias.png}
	\caption{Costo de oportunidad de las limitaciones del sistema de recomendación actual. Hay un par de errores menores en esta gráfica que corregiré en la versión final.}
	\label{fig:Limitaciones_Sistema_Actual}
\end{figure}

Como se puede visualizar en la figura anterior, la implementación del sistema actual está desaprovechando un mayor número de clientes potenciales, lo que se traduce en una pérdida significativa de ingresos para la empresa. Esta situación resalta la necesidad de mejorar el sistema de recomendación para captar mejor las oportunidades de venta y maximizar los beneficios comerciales. Al implementar un sistema de recomendación basado en dominios cruzados (CDRS) que utilice datos de las diferentes unidades de negocio de Falabella, se espera reducir este costo de oportunidad al aumentar la precisión y relevancia de las recomendaciones para clientes nuevos en situación de \enquote{cold-start}. Esto permitirá a la empresa aprovechar al máximo su base de clientes y mejorar su desempeño en el mercado altamente competitivo.

¿Cuáles son las causas de la existencia de de este problema u oportunidad?
No sé si acá hay que justificar las causas en base a lo determinado por FTC o si es (así como, \enquote{las causas del problema son Tangananica y Tangananá}, como lo que está en los primeros dos párrafos), más bien, justificar las causas en base a la literatura académica.

Explique cómo la memoria ayudará a abordar este problema.
No me queda claro a qué se refieren con \enquote{memoria} en este contexto. Pero asumo que se refieren a la tesis en sí misma.

En este contexto, la memoria de la tesis abordará el problema del \enquote{cold-start} en sistemas de recomendación mediante el desarrollo de un sistema basado en dominios cruzados (CDRS) que aproveche los datos de las diferentes unidades de negocio del grupo Falabella. Para ello, se desarrolló un sistema de agentes de modelos de lenguaje grande (LLM) para la tarea de relacionar distintos niveles de categorías que poseen los productos (entiéndase los niveles de categorías como, por ejemplo, \enquote{categoría: Hombre, sub-categoría: Zapatos} para la unidad de negocio de Falabella Retail), donde se agrupan y etiquetan las categorías que sean similares entre las distintas unidades de negocio. Estos nuevos atributos se pueden utilizar como características adicionales en el sistema de recomendación, permitiendo generar un modelo de sistema de recomendación que sea capaz de realizar recomendaciones más personalizadas y precisas para clientes nuevos en situación de \enquote{cold-start}, con el fin de que éstos se puedan convertir en clientes recurrentes.
% Chapter 1

\chapter{Estado del arte} % Main chapter title
\label{sec:Estado Arte} % For referencing the chapter elsewhere, use

El estado del arte debería al menos responder a las siguientes preguntas:
\begin{itemize}
	\item ¿Cómo se ha enfrentado o se está enfrentando este problema u oportunidad en la literatura académica? 
	\item ¿Existen proyectos en desarrollo en la misma línea de investigación? 
	\item ¿Qué soluciones y métodos ya existen? 
\end{itemize}

Considere información nacional e internacional actualizada sobre publicaciones, proyectos tecnológicos y líneas de investigación y desarrollo en empresas u otro tipo de organizaciones. 

Dependiendo del tipo de tesis, en algunos casos se hace una búsqueda de patentes y de otros registros de propiedad intelectual, a nivel nacional e internacional, relativos al problema/oportunidad que se piensa abordar e indicar los resultados de la búsqueda. 

Realice una búsqueda y análisis de estándares, normas y reglamentaciones, tanto nacionales como extranjeras e internacionales, pertinentes y aplicables al tema del proyecto. 

Tiene 3 o 4 subsecciones. Son como 4\/6 planas. 
\begin{itemize}
	\item Sistemas de recomendación.
	\item \enquote{cold-start} problem.
	\item Atributos/features en sistemas de recomendación.
	\item Modelos de lenguaje grande en sistemas de recomendación.
\end{itemize}
\chapter{Hipótesis y objetivos} % Main chapter title
\label{sec:Hipotesis} % For referencing the chapter elsewhere, use

\begin{quote}
  \textit{El uso de Modelos de Lenguaje Grande (LLM) para generar \enquote{super-categorías} unifica semánticamente los catálogos de productos de distintas unidades de negocio, mejorando el rendimiento de los sistemas de recomendación en escenarios de \enquote{cold-start} de usuario.}
\end{quote}

Esta investigación busca explorar la capacidad de los Modelos de Lenguaje Grande (LLM) para mitigar el problema del \enquote{cold-start} en sistemas de recomendación mediante la creación de una taxonomía unificada que agrupe categorías similares de productos entre diferentes dominios comerciales. Esto permitiría al sistema de recomendación transferir conocimiento entre unidades de negocio, mejorando la calidad de las recomendaciones para usuarios en situación de \enquote{cold-start}.

\section{Objetivos}

  \subsection{Objetivo General}
  Desarrollar y validar una metodología para la mitigación del problema del \enquote{cold-start} en sistemas de recomendación mediante la unificación semántica de dominios comerciales heterogéneos, utilizando un sistema multi-agente basado en Modelos de Lenguaje Grande (LLM) para la generación de taxonomías transversales (\enquote{super-categorías}).

  \subsection{Objetivos Específicos}
  \begin{enumerate}
    \item Consolidar un marco de datos unificados de los catálogos de productos y los historiales de interacción de las unidades de negocio del grupo Falabella (Falabella Retail, Tottus y Sodimac) para crear un dataset estandarizado.
    \item Desarrollar un mecanismo de alineación semántica autónomo mediante un sistema multi-agente basado en LLM capaz de generar y validar \enquote{super-categorías} a partir de las categorías originales de los catálogos de productos.
    \item Integrar las \enquote{super-categorías} generadas como características latentes adicionales a los datasets de entrenamiento de un sistema de recomendación.
    \item Evaluar el impacto de la inclusión de \enquote{super-categorías} en el rendimiento del sistema de recomendación, especialmente en escenarios de \enquote{cold-start} de usuario, mediante métricas cuantitativas como NDCG y Recall.
  \end{enumerate}
\chapter{Metodología} % Main chapter title
\label{sec:Metodologia} % For referencing the chapter elsewhere, use

La presente metodología se estructura en torno a la hipótesis de que la integración de conocimiento semántico externo, a través de Modelos de Lenguaje Grande (LLM), puede mitigar eficazmente el problema del \enquote{cold-start} de usuarios en sistemas de recomendación, en especial cuando se trata de dominios comerciales heterogéneos. A diferencia de los enfoques tradicionales que dependen exclusivamente de la coincidencia exacta de identificadores o de interacciones históricas, esta propuesta busca construir un \enquote{puente semántico} latente entre unidades de negocio dispares (Falabella Retail, Sodimac y Tottus) mediante la creación de \enquote{super-categorías} que unifiquen conceptualmente los catálogos de productos. Este proceso se llevará a cabo mediante un sistema multi-agente basado en LLM, que permitirá generar y validar estas nuevas categorías de manera iterativa y autónoma.

La literatura reciente respalda este cambio de paradigma. \cite{liang2025taxonomy} destacan que el uso de taxonomías guiadas por LLM permiten realizar recomendaciones \enquote{zero-shot} efectivas al estructurar información no organizada, superando las limitaciones de los métodos basados en filtrado colaborativo o contenido puro. Asimismo, investigaciones recientes sobre la capacidad de transferencia de los LLM señalan que estos modelos pueden actuar como razonadores de dominios cruzados, infiriendo preferencias de usuarios en un dominio objetivo basándose en comportamientos de un dominio fuente, incluso sin solapamiento de ítems \cite{liu2025uncovering, krishna2024cross}. A continuación se detalla un esquema general de la metodología propuesta en la Figura \ref{fig:metodologia} y un desglose paso a paso en la Figura \ref{fig:metodologia-pasos}.

\begin{figure}[th]
  \centering
  \includegraphics[width=0.8\textwidth]{Figures/Resumen-grafico.png}
  \caption{Esquema general de la metodología propuesta.}
  \label{fig:metodologia}
\end{figure}

\begin{figure}[th]
  \centering
  \includegraphics[width=\textwidth]{Figures/metodologia-revisada.pdf}
  \caption{Esquema paso a paso de la metodología propuesta.}
  \label{fig:metodologia-pasos}
  
\end{figure}

\begin{enumerate}
  \item \textbf{Recopilación y procesamiento de los datos:} Obtener y estandarizar los catálogos de productos e historiales de interacción de las unidades de negocio del grupo Falabella (Falabella Retail, Tottus y Sodimac) para crear un dataset unificado.
  \item \textbf{Generación de \enquote{super-categorías}:} Implementar un sistema multi-agente basado en LLM para crear y validar \enquote{super-categorías} a partir de las categorías originales de los catálogos de productos.
  \item \textbf{Creación del dataset extendido:} Incorporar las \enquote{super-categorías} generadas como características latentes adicionales en los datasets de entrenamiento del sistema de recomendación.
  \item \textbf{Entrenamiento y fine-tuning del sistema de recomendación:} Ajustar el modelo de recomendación utilizando el dataset extendido.
  \item \textbf{Evaluación del rendimiento del sistema de recomendación:} Medir la precisión, recall y otras métricas relevantes para evaluar la efectividad del sistema en comparación con el sistema actual.
  \item \textbf{Consolidación de los resultados:} Analizar y documentar los hallazgos obtenidos durante la evaluación para futuras mejoras.
\end{enumerate}


\section{Recopilación de datos del grupo Falabella}

  La primera etapa de la metodología propuesta consiste en la recopilación y preprocesamiento de los datos provenientes de las distintas unidades de negocio del grupo Falabella. Esto incluye la obtención de los catálogos de productos y los historiales de interacción de los usuarios con dichos productos. Estos datos serán estandarizados para crear un dataset unificado que servirá como base para las etapas posteriores.

  En particular, se extraer los catálogos de productos de las tablas de maestro de productos de cada unidad de negocio, asegurando de obtener todos los atributos pertenecientes a cada producto, como su categoría, descripción, precio, entre otros. Luego, para obtener los historiales de interacción, se cruzarán las tablas de transacciones de los usuarios con los catálogos de productos, generando un dataset que refleje las interacciones de los usuarios con los productos disponibles en cada unidad de negocio. Este dataset unificado permitirá analizar las similitudes y diferencias entre las categorías de productos de las distintas unidades de negocio, facilitando la generación de \enquote{super-categorías} en la siguiente etapa.

  En esta etapa, también se llevará a cabo un proceso de limpieza y normalización de los datos para garantizar su calidad y consistencia. Esto incluirá la eliminación de duplicados, el manejo de valores faltantes y una mejor comprensión sobre la estructura de los datos disponibles.

\section{Generación de \enquote{super-categorías}}

  En esta etapa, se implementará un sistema multi-agente basado en Modelos de Lenguaje Grande (LLM) para la generación y validación de \enquote{super-categorías} a partir de las categorías originales de los catálogos de productos. Este sistema estará compuesto por tres agentes, cada uno con una función específica en el proceso de generación de \enquote{super-categorías}, los cuales se describen a continuación:

  \subsection{El agente generador}
    Este agente constituye el punto de partida y el principal actor del sistema multi-agente. Su función principal es identificar y agrupar categorías similares entre las distintas unidades de negocio, cuyos grupos serán etiquetados con una nueva \enquote{super-categoría} que mejor represente el conjunto. Para ello, este agente utiliza la capacidad de razonamiento de un LLM para identificar equivalencias conceptuales entre etiquetas dispares (por ejemplo, asociar \enquote{Cunas} de una unidad de negocio con \enquote{Ropa de bebé} de otra). Este proceso se realizará iterativamente por cada nivel de la jerarquía de categorías, comenzando desde el nivel más general hasta llegar al nivel más específico.

    Luego de generar una propuesta inicial de \enquote{super-categorías}, el agente empaqueta cada grupo de categorías hijas junto con su etiqueta propuesta en un formato estructurado (por ejemplo, JSON) y lo envía al Agente Validador para su evaluación.

  \subsection{El agente validador}
    Este agente opera como un filtro de calidad crítico dentro del flujo de trabajo. Su función principal es evaluar la coherencia semántica y la precisión de las \enquote{super-categorías} propuestas por el Agente Generador, sin intervenir en la creación de nuevas etiquetas.

    Utilizando una instancia de LLM configurada con instrucciones de auditoría estricta, el agente analiza cada par \textit{(super-categoría, categorías hijas)} bajo criterios predefinidos de:
    \begin{itemize}
      \item \textbf{Homogeneidad:} Verifica que todas las sub-categorías agrupadas compartan una relación semántica lógica.
      \item \textbf{Representatividad:} Asegura que la etiqueta de la super-categoría describa fielmente al conjunto sin ser excesivamente genérica ni restrictiva.
      \item \textbf{Detección de Alucinaciones:} Identifica agrupaciones erróneas o etiquetas inventadas que no corresponden a la taxonomía comercial real.
    \end{itemize}
   
   El resultado de esta etapa es un veredicto binario (Aprobado/Rechazado) acompañado de una justificación estructurada. En caso de rechazo, este reporte de error se envía al Agente Retroalimentador, quien será el encargado de interpretar las fallas e instruir al Generador en la siguiente iteración.

  \subsection{El agente retroalimentador}
    La función principal de este agente se activa únicamente cuando el agente validador rechaza una propuesta. El agente retroalimentador recibe el reporte de error o inconsistencia generado por el validador y lo procesa para construir un nuevo \textit{prompt} o conjunto de instrucciones refinadas.

    A diferencia de un simple reintento, este agente contextualiza el error (por ejemplo, si la \enquote{super-categoría} fue muy genérica o semánticamente incorrecta) y guía al agente generador para que su siguiente iteración sea más precisa. Este ciclo de retroalimentación se repite hasta que la propuesta cumple con los estándares de validación o hasta alcanzar un número máximo de iteraciones, garantizando así que las \enquote{super-categorías} resultantes no solo sean válidas, sino que estén optimizadas semánticamente para la tarea de unificación de dominios.

\section{Creación del dataset extendido}

  Una vez obtenidas y validadas las \enquote{super-categorías}, la tercera etapa se centra en la integración de esta nueva información en los datos de entrenamiento para el sistema de recomendación. El objetivo es enriquecer la representación de los ítems y usuarios mediante la incorporación de estas características latentes.

  Para lograr esto, se transformará la salida estructurada del sistema multi-agente (por ejemplo, en formato JSON) en un formato tabular compatible con los datasets de entrenamiento del sistema de recomendación, con el fin de permitir el cruzamiento de la tabla de \enquote{super-categorías} con las tablas de productos e historiales de interacción. Esto implicará la creación de nuevas columnas en las tablas de productos que reflejen las \enquote{super-categorías} asignadas a cada producto.

\section{Entrenamiento y fine-tuning del sistema de recomendación}

  Con el dataset unificado y enriquecido, se procederá a la fase de modelado. En esta etapa se seleccionará y entrenará el algoritmo de recomendación (por ejemplo, BigQuery ML) capaz de manejar características de dominios cruzados.

  El proceso de entrenamiento se dividirá en dos sub-etapas clave:
  \begin{itemize}
    \item \textbf{Entrenamiento inicial:} Se entrenará el modelo de recomendación utilizando el dataset extendido que incluye las \enquote{super-categorías}, permitiendo al modelo aprender patrones de interacción entre usuarios y productos a través de los dominios comerciales.
    \item \textbf{Fine-tuning:} Posteriormente, se realizará un ajuste fino del modelo para optimizar su rendimiento específicamente en escenarios de \enquote{cold-start} de usuario. Esto implicará la evaluación continua del modelo durante el entrenamiento, ajustando hiperparámetros y técnicas de regularización para maximizar la capacidad del sistema para generar recomendaciones precisas incluso cuando la información del usuario es limitada.
  \end{itemize}

\section{Evaluación del rendimiento del sistema de recomendación}

  Para validar la hipótesis de investigación, se llevará a cabo una evaluación cuantitativa rigurosa comparando el desempeño del nuevo modelo frente al sistema base (sin \enquote{super-categorías}).

  La estrategia de evaluación consistirá en dividir el dataset en conjuntos de entrenamiento y prueba, aislando específicamente a usuarios que simulen una situación de \textit{cold-start} en el dominio objetivo. Se utilizarán métricas de ranking estándar en la industria, tales como:
  \begin{itemize}
    \item \textbf{NDCG@K (Normalized Discounted Cumulative Gain):} Para evaluar la calidad del ordenamiento de las recomendaciones.
    \item \textbf{Recall@K:} Para medir la capacidad del sistema de recuperar ítems relevantes dentro de las primeras $K$ sugerencias.
  \end{itemize}
  El éxito de la metodología se determinará en función de la mejora porcentual (\textit{lift}) obtenida en estas métricas respecto al baseline.

\section{Consolidación de los resultados}

  Finalmente, la última etapa consiste en la recopilación, análisis y documentación de los hallazgos obtenidos durante la fase experimental. Se generarán reportes que no solo muestren las métricas de rendimiento, sino que también analicen cualitativamente la calidad de las \enquote{super-categorías} generadas por los agentes LLM. Esta consolidación permitirá identificar las fortalezas de la metodología propuesta, así como las limitaciones y áreas de oportunidad para futuras líneas de investigación en sistemas de recomendación \textit{cross-domain}.
\chapter{Resultados y análisis} % Main chapter title
\label{sec:resultados} % For referencing the chapter elsewhere, use

Para cada objetivo específico, describa los resultados obtenidos, analícelos y explíquelos claramente.
\chapter{Conclusiones} % Main chapter title
\label{sec:conclusiones} % For referencing the chapter elsewhere, use

Conclusiones de su trabajo: ¿se cumplieron los objetivos? ¿se validó la hipótesis? Debería además incluir líneas de investigación futura.

% Especifica el estilo de la bibliografía
\bibliographystyle{apacite} 
% \bibliographystyle{ieeetr}

% Llama a tu archivo .bib
\bibliography{citas}

%----------------------------------------------------------------------------------------
%	THESIS CONTENT - APPENDICES
%----------------------------------------------------------------------------------------

\appendix % Cue to tell LaTeX that the following "chapters" are Appendices

% Include the appendices of the thesis as separate files from the Appendices folder
% Uncomment the lines as you write the Appendices

% Appendix Template

\chapter{Título del apéndice} % Main appendix title

\label{AppendixX} % Change X to a consecutive letter; for referencing this appendix elsewhere, use \ref{AppendixX}

Write your Appendix content here.

%----------------------------------------------------------------------------------------
%	BIBLIOGRAPHY
%----------------------------------------------------------------------------------------


%\printbibliography[heading=bibintoc]
%\printbibliography

%----------------------------------------------------------------------------------------

\end{document}
% Chapter 1

\chapter{Estado del arte} % Main chapter title
\label{sec:Estado Arte} % For referencing the chapter elsewhere, use

El estado del arte debería al menos responder a las siguientes preguntas:
\begin{itemize}
	\item ¿Cómo se ha enfrentado o se está enfrentando este problema u oportunidad en la literatura académica? 
	\item ¿Existen proyectos en desarrollo en la misma línea de investigación? 
	\item ¿Qué soluciones y métodos ya existen? 
\end{itemize}

Considere información nacional e internacional actualizada sobre publicaciones, proyectos tecnológicos y líneas de investigación y desarrollo en empresas u otro tipo de organizaciones. 

Dependiendo del tipo de tesis, en algunos casos se hace una búsqueda de patentes y de otros registros de propiedad intelectual, a nivel nacional e internacional, relativos al problema/oportunidad que se piensa abordar e indicar los resultados de la búsqueda. 

Realice una búsqueda y análisis de estándares, normas y reglamentaciones, tanto nacionales como extranjeras e internacionales, pertinentes y aplicables al tema del proyecto.

\newpage

\section{Sistemas de recomendación}

	Esta sección presenta una revisión detallada sobre los sistemas de recomendación, abarcando su historia, tipos y enfoques modernos.

	\subsection{Historia de los sistemas de recomendación}
	En esta subsección se presenta una breve revisión histórica sobre los sistemas de recomendación, desde sus inicios hasta la actualidad. Se destacan los hitos más importantes y las tecnologías que han impulsado su evolución.

	\subsection{Tipos de sistemas de recomendación}
	En esta subsección se describen los diferentes tipos de sistemas de recomendación, incluyendo filtrado colaborativo, filtrado basado en contenido y enfoques híbridos. Se discuten las ventajas y desventajas de cada tipo, así como su aplicabilidad en diferentes contextos.

		\subsubsection{Filtrado colaborativo}

		\subsubsection{Filtrado basado en contenido}
		
		\subsubsection{Sistemas basados en aprendizaje profundo (DLRS)}
		
		\subsubsection{Sistemas basados en dominios cruzados (CDRS)}

\section{El problema de \enquote{cold-start}}

	Esta sección aborda el problema de \enquote{cold-start} en los sistemas de recomendación, que se refiere a la dificultad de hacer recomendaciones precisas para nuevos usuarios o ítems sin historial previo.

\section{Modelos de lenguaje grande (LLM) en sistemas de recomendación}

	Esta sección explora el uso de modelos de lenguaje grande (LLM) en sistemas de recomendación, analizando su potencial para mejorar la calidad de las recomendaciones y abordar desafíos como el problema de \enquote{cold-start}.

Breve conclusión del estado del arte, identificando las brechas en la literatura y justificando la necesidad de la investigación propuesta en la tesis.

% \section{Atributos/features en sistemas de recomendación}
\newpage

Esto es otra estructura posible:

\section{El problema del \enquote{cold-start} en sistemas de recomendación}
	Esta sección aborda el problema de \enquote{cold-start} en los sistemas de recomendación, que se refiere a la dificultad de hacer recomendaciones precisas para nuevos usuarios o ítems sin historial previo.

	\subsection{Sistemas de recomendación y sus implementaciones más comunes}
		En esta subsección se describen los diferentes tipos de sistemas de recomendación, incluyendo filtrado colaborativo, filtrado basado en contenido y enfoques híbridos. Se discute por qué estos enfoques enfrentan desafíos significativos cuando se trata de usuarios o ítems nuevos sin historial previo.

	\subsection{El problema del \enquote{cold-start}}
		En esta subsección se profundiza en el problema del \enquote{cold-start}, explicando sus causas y consecuencias en los sistemas de recomendación. Se presentan ejemplos de situaciones en las que este problema es particularmente relevante, como en plataformas de comercio electrónico con múltiples unidades de negocio.

\section{Cross-Domain Recommender Systems (CDRS) como solución al \enquote{cold-start}}
	Esta sección explora el enfoque de recomendación de cross-domain (CDRS) como una solución potencial al problema del \enquote{cold-start}. Se discuten las técnicas y metodologías utilizadas en CDRS para transferir conocimiento entre dominios relacionados y mejorar la calidad de las recomendaciones para usuarios o ítems nuevos.

	\subsection{Definición y conceptos clave de CDRS}
		En esta subsección se define el concepto de recomendación de cross-domain (CDRS) y se explican los conceptos clave asociados, como la transferencia de conocimiento entre dominios y la creación de perfiles de usuario enriquecidos.

	\subsection{Técnicas y metodologías en CDRS}
		En esta subsección se describen las técnicas y metodologías utilizadas en CDRS, incluyendo enfoques basados en modelos, enfoques basados en memoria y enfoques híbridos. Se discuten las ventajas y desventajas de cada técnica, así como su aplicabilidad en diferentes contextos.

	\subsection{El principal desafío de los CDRS}
		En esta subsección se aborda el principal desafío de los CDRS, que es la identificación y vinculación de ítems similares entre dominios diferentes. Presentar claramente el \enquote{knowledge gap}.

\section{Modelos de Lenguaje Grande (LLM) en sistemas de recomendación}
	Esta sección explora el uso de modelos de lenguaje grande (LLM) en sistemas de recomendación, analizando su potencial para mejorar la calidad de las recomendaciones y abordar desafíos como el problema de \enquote{cold-start}.

	\subsection{LLM como sistemas de recomendación}
		En esta subsección se describe cómo los modelos de lenguaje grande (LLM) pueden ser utilizados como sistemas de recomendación, aprovechando su capacidad para comprender y generar texto natural. Se presentan ejemplos de aplicaciones exitosas de LLM en sistemas de recomendación.

	\subsection{LLM como extractores de características}
		En esta subsección se explora el uso de LLM como extractores de características para enriquecer los perfiles de usuario y los ítems en sistemas de recomendación. Se discuten las técnicas utilizadas para extraer características relevantes y su impacto en la calidad de las recomendaciones.

	\subsection{LLM para tareas de \textit{mapeo} y unificación de dominios de información}
		En esta subsección se analiza cómo los LLM pueden ser utilizados para tareas de \textit{mapeo} y unificación de dominios de información en sistemas de recomendación. Se presentan técnicas para vincular ítems similares entre dominios diferentes, abordando el desafío principal de los CDRS.

Breve conclusión del estado del arte, identificando las brechas en la literatura y justificando la necesidad de la investigación propuesta en la tesis.
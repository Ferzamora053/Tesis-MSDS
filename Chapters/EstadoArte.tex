% Chapter 1

\chapter{Estado del arte} % Main chapter title
\label{sec:Estado Arte} % For referencing the chapter elsewhere, use

El estado del arte debería al menos responder a las siguientes preguntas:
\begin{itemize}
	\item ¿Cómo se ha enfrentado o se está enfrentando este problema u oportunidad en la literatura académica? 
	\item ¿Existen proyectos en desarrollo en la misma línea de investigación? 
	\item ¿Qué soluciones y métodos ya existen? 
\end{itemize}

Considere información nacional e internacional actualizada sobre publicaciones, proyectos tecnológicos y líneas de investigación y desarrollo en empresas u otro tipo de organizaciones. 

Dependiendo del tipo de tesis, en algunos casos se hace una búsqueda de patentes y de otros registros de propiedad intelectual, a nivel nacional e internacional, relativos al problema/oportunidad que se piensa abordar e indicar los resultados de la búsqueda. 

Realice una búsqueda y análisis de estándares, normas y reglamentaciones, tanto nacionales como extranjeras e internacionales, pertinentes y aplicables al tema del proyecto. 

Tiene 3 o 4 subsecciones. Son como 4/6 planas. 
\begin{itemize}
	\item Sistemas de recomendación.
	\item \enquote{cold-start} problem.
	% \item Atributos/features en sistemas de recomendación.
	\item Modelos de lenguaje grande en sistemas de recomendación.
\end{itemize}

\newpage

\section{Sistemas de recomendación}

	Esta sección presenta una revisión detallada sobre los sistemas de recomendación, abarcando su historia, tipos y enfoques modernos.

	\subsection{Historia de los sistemas de recomendación}
	En esta subsección se presenta una breve revisión histórica sobre los sistemas de recomendación, desde sus inicios hasta la actualidad. Se destacan los hitos más importantes y las tecnologías que han impulsado su evolución.

	\subsection{Tipos de sistemas de recomendación}
	En esta subsección se describen los diferentes tipos de sistemas de recomendación, incluyendo filtrado colaborativo, filtrado basado en contenido y enfoques híbridos. Se discuten las ventajas y desventajas de cada tipo, así como su aplicabilidad en diferentes contextos.

		\subsubsection{Filtrado colaborativo}

		\subsubsection{Filtrado basado en contenido}
		
		\subsubsection{Sistemas basados en aprendizaje profundo (DLRS)}
		
		\subsubsection{Sistemas basados en dominios cruzados (CDRS)}

\section{El problema de \enquote{cold-start}}

	Esta sección aborda el problema de \enquote{cold-start} en los sistemas de recomendación, que se refiere a la dificultad de hacer recomendaciones precisas para nuevos usuarios o ítems sin historial previo.

\section{Modelos de lenguaje grande (LLM) en sistemas de recomendación}

	Esta sección explora el uso de modelos de lenguaje grande (LLM) en sistemas de recomendación, analizando su potencial para mejorar la calidad de las recomendaciones y abordar desafíos como el problema de \enquote{cold-start}.

% \section{Atributos/features en sistemas de recomendación}
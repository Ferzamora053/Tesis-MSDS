% Chapter 1
\chapter{Introducción} % Main chapter title
\label{sec:Introduccion} % For referencing the chapter elsewhere, use

\textbf{OMITIR ESTA SECCION PARA EL CASO DEL MAGISTER EN CIENCIAS DE LA INGENIER\'IA}

De forma de acotar la propuesta lo máximo posible, este capítulo debería responder contener al menos los siguientes puntos:
\begin{itemize}
\item Una breve descripción del lugar donde está generada la oportunidad.
\item El contexto u oportunidad de investigación.
\item La importancia y motivación de la investigación y cómo su trabajo es relevante.
\end{itemize}

\newpage
Los sistemas de recomendación tienen una gran importancia en la actualidad, puesto que permiten personalizar la experiencia del usuario en diversas plataformas digitales. Según Google\footnote{\href{https://developers.google.com/machine-learning/glossary}{Fuente.}}, estos son sistemas que seleccionan para cada usuario un conjunto relativamente pequeño de elementos deseables de un gran conjunto de posibles opciones.

Estos sistemas utilizan como entrada grandes volúmenes de datos generados por las interacciones de los usuarios con los productos o servicios ofrecidos. A través del análisis de estos datos, los sistemas de recomendación pueden identificar patrones y preferencias individuales, lo que les permite sugerir productos o servicios que se ajusten a los intereses específicos de cada usuario.

Entonces, ¿qué sucede cuando los datos son escasos o inexistentes? Este es el problema conocido como \enquote{cold start} o \enquote{arranque en frío}. En tales situaciones, los sistemas de recomendación enfrentan dificultades para generar sugerencias precisas debido a la falta de información suficiente sobre las preferencias del usuario o las características del producto. Además, es importante diferenciar entre los distintos tipos de \enquote{cold start}, como el \enquote{cold start} de usuario, donde un nuevo usuario no tiene historial previo, y el \enquote{cold start} de ítem, donde un nuevo producto carece de interacciones previas, puesto que ambos tipos presentan desafíos únicos que requieren enfoques específicos para su mitigación \cite{10339320}, y, por tanto, el enfoque de esta tesis se centrará en el \enquote{cold start} de usuarios.

Este desafío puede ser particularmente agudo en grandes conglomerados de retail que operan con múltiples unidades de negocio (como el grupo Falabella, donde tienen unidades de negocio en los sectores de retail, mejoramiento del hogar, supermercado y otros). En este escenario, un cliente puede tener un amplio historial de compras en una unidad, pero ser completamente "nuevo" para otra. Esta desconexión, causada por catálogos de productos que operan en silos, se traduce en una pérdida significativa de oportunidad comercial y en una experiencia de usuario fragmentada. Por lo tanto, encontrar una forma escalable de transferir el conocimiento del cliente entre estos dominios (unidades de negocio) es una de las principales motivaciones para la industria y la investigación.

Hablar sobre que se ha hecho para mitigar este problema, y que aún existen desafíos abiertos en el área. Utilizar citas de papers.


Por lo tanto, el objetivo de esta tesis es investigar y desarrollar una metodología que permita mitigar el problema del \enquote{cold start} en sistemas de recomendación, utilizando técnicas de dominios de información interrelacionados mediante la generación de \enquote{super-categorías} con modelos de lenguaje grande (LLM) para la vinculación de productos entre distintas unidades de negocio del grupo Falabella.

Para ello, se trabajará en conjunto con la empresa Falabella Tecnología Corporativa (FTC), que cuenta con una amplia base de datos de usuarios y productos en sus plataformas de comercio electrónico, así como distintas unidades de negocio que pueden proporcionar información complementaria para abordar el problema del \enquote{cold start} \cite{L__2012}.
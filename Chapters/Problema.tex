% Chapter 1
\chapter{Definición del problema u oportunidad} % Main chapter title
\label{sec:problema} % For referencing the chapter elsewhere, use

Este capítulo debe contener la exposición general del problema. Con respecto al problema se debe responder las siguientes preguntas:
\begin{itemize}
	\item	¿Cuál es el problema u oportunidad abordado por el proyecto? ¿Es posible cuantificarlo?
	\item	¿Cuáles son las causas de la existencia de este problema u oportunidad? Haga referencia a publicaciones y/u otros antecedentes que validen estas causas.
	\item Explique cómo la memoria ayudará a abordar este problema.
	
\end{itemize}

\section{Ejemplos de latex}

Sed ullamcorper quam eu nisl interdum at interdum enim egestas. Aliquam placerat justo sed lectus lobortis ut porta nisl porttitor. Vestibulum mi dolor, lacinia molestie gravida at, tempus vitae ligula. Donec eget quam sapien, in viverra eros. Donec pellentesque justo a massa fringilla non vestibulum metus vestibulum. Vestibulum in orci quis felis tempor lacinia. Vivamus ornare ultrices facilisis. Ut hendrerit volutpat vulputate. Morbi condimentum venenatis augue, id porta ipsum vulputate in. Curabitur luctus tempus justo. Vestibulum risus lectus, adipiscing nec condimentum quis, condimentum nec nisl. Aliquam dictum sagittis velit sed iaculis. Morbi tristique augue sit amet nulla pulvinar id facilisis ligula mollis. Nam elit libero, tincidunt ut aliquam at, molestie in quam. Aenean rhoncus vehicula hendrerit.

\begin{figure}[th]
	\centering
	\includegraphics[width=10cm]{Figures/Ejemplo}
	\caption{Comparación de la participación de mercado entre mutuales, a enero 2020.}
	\label{fig:Ejemplo}
\end{figure}

% Citando Figura~\ref{fig:Ejemplo} de la sección~\ref{sec:problema}

Lista de items
\begin{itemize}
	\item \textbf{Componente A:} Sed ullamcorper quam eu nisl interdum at interdum enim egestas. Aliquam placerat justo sed lectus lobortis ut porta nisl porttitor. Vestibulum mi dolor, lacinia molestie gravida at, tempus vitae ligula. Donec eget quam sapien, in viverra eros. Donec pellentesque justo a massa fringilla non vestibulum metus vestibulum. Vestibulum in orci quis felis tempor lacinia. Vivamus ornare ultrices facilisis. Ut hendrerit volutpat vulputate. Morbi condimentum venenatis augue, id porta ipsum vulputate in.
	\item \textbf{Componente B:} Sed ullamcorper quam eu nisl interdum at interdum enim egestas. Aliquam placerat justo sed lectus lobortis ut porta nisl porttitor. Vestibulum mi dolor, lacinia molestie gravida at, tempus vitae ligula. Donec eget quam sapien, in viverra eros. Donec pellentesque justo a massa fringilla non vestibulum metus vestibulum. Vestibulum in orci quis felis tempor lacinia. Vivamus ornare ultrices facilisis. Ut hendrerit volutpat vulputate. Morbi condimentum venenatis augue, id porta ipsum vulputate in.
\end{itemize}

Lista enumerada
\begin{enumerate}
	\item \textbf{Componente A:} Sed ullamcorper quam eu nisl interdum at interdum enim egestas. Aliquam placerat justo sed lectus lobortis ut porta nisl porttitor. Vestibulum mi dolor, lacinia molestie gravida at, tempus vitae ligula. Donec eget quam sapien, in viverra eros. Donec pellentesque justo a massa fringilla non vestibulum metus vestibulum. Vestibulum in orci quis felis tempor lacinia. Vivamus ornare ultrices facilisis. Ut hendrerit volutpat vulputate. Morbi condimentum venenatis augue, id porta ipsum vulputate in.
	\item \textbf{Componente B:} Sed ullamcorper quam eu nisl interdum at interdum enim egestas. Aliquam placerat justo sed lectus lobortis ut porta nisl porttitor. Vestibulum mi dolor, lacinia molestie gravida at, tempus vitae ligula. Donec eget quam sapien, in viverra eros. Donec pellentesque justo a massa fringilla non vestibulum metus vestibulum. Vestibulum in orci quis felis tempor lacinia. Vivamus ornare ultrices facilisis. Ut hendrerit volutpat vulputate. Morbi condimentum venenatis augue, id porta ipsum vulputate in.
\end{enumerate}

% Citas a libro \cite{ejemploLibro} o paper~\cite{ejemploPaper}. Usted puede agregar otros tipos como journal o techincal report
\newpage

Falabella Tecnología Corporativa actualmente posee un sistema de recomendación (Implicit Collaborative Filtering - Matrix Factorization) desarrollado en BigQuery ML (BQML) que presenta varias limitaciones importantes para el negocio: es demasiado simple para abordar la complejidad del comportamiento del cliente, tiene un bajo nivel de personalización y utiliza únicamente el identificador del cliente como dato de entrada principal, desaprovechando datos relevantes como características demográficas, comportamientos históricos y segmentaciones internas.

Estas limitaciones restrigen la capacidad del sistema para generar recomendaciones personalizadas que maximicen la efectividad de las campañas de marketing. Por lo tanto, es crucial desarrollar un nuevo enfoque que integre estos datos adicionales para mejorar la precisión y relevancia de las recomendaciones, alineándolas mejor con las necesidades y preferencias individuales de los clientes. Esto permitirá optimizar las estrategias de marketing, aumentar la satisfacción del cliente y, en última instancia, mejorar los resultados comerciales de Falabella Tecnología Corporativa.

Me falta un puente para conectar lo propuesto en el archivo de la propuesta del proyecto de tesis (los párrafos anteriores) con el sistema de agentes desarrollado y lo relacionado a los Cross-Domain Recommender Systems (CDRS).

\begin{figure}[th]
	\centering
	\includegraphics[width=\textwidth]{Figures/grafica Matias.png}
	\caption{Costo de oportunidad de las limitaciones del sistema de recomendación actual. Hay un par de errores menores en esta gráfica que corregiré en la versión final.}
	\label{fig:Limitaciones_Sistema_Actual}
\end{figure}

Como se puede visualizar en la figura anterior, la implementación del sistema actual está desaprovechando un mayor número de clientes potenciales, lo que se traduce en una pérdida significativa de ingresos para la empresa. Esta situación resalta la necesidad de mejorar el sistema de recomendación para captar mejor las oportunidades de venta y maximizar los beneficios comerciales. Al implementar un sistema de recomendación basado en dominios cruzados (CDRS) que utilice datos de las diferentes unidades de negocio de Falabella, se espera reducir este costo de oportunidad al aumentar la precisión y relevancia de las recomendaciones para clientes nuevos en situación de \enquote{cold-start}. Esto permitirá a la empresa aprovechar al máximo su base de clientes y mejorar su desempeño en el mercado altamente competitivo.